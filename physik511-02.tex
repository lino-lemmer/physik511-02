% Für Seitenformatierung

\documentclass[DIV=15]{scrartcl}

% Zeilenumbrüche

\parindent 0pt
\parskip 6pt

% Für deutsche Buchstaben und Synthax

\usepackage[ngerman]{babel}

% Für Auflistung mit speziellen Aufzählungszeichen

\usepackage{paralist}

% zB für \del, \dif und andere Mathebefehle

\usepackage{amsmath}
\usepackage{commath}
\usepackage{amssymb}

% für nicht kursive griechische Buchstaben

\usepackage{txfonts}

% Für \SIunit[]{} und \num in deutschem Stil

\usepackage[output-decimal-marker={,}]{siunitx}
\usepackage[utf8]{inputenc}

% Für \sfrac{}{}, also inline-frac

\usepackage{xfrac}

% Für Einbinden von pdf-Grafiken

\usepackage{graphicx}

% Umfließen von Bildern

\usepackage{floatflt}

% Für Links nach außen und innerhalb des Dokumentes

\usepackage{hyperref}

% Für weitere Farben

\usepackage{color}

% Für Streichen von z.B. $\rightarrow$

\usepackage{centernot}

% Für Befehl \cancel{}

\usepackage{cancel}

% Für Layout von Links

\hypersetup{
	citecolor=black,
	colorlinks=true,
	linkcolor=black,
	urlcolor=blue,
}

% Verschiedene Mathematik-Hilfen

\newcommand \e[1]{\cdot10^{#1}}
\newcommand\p{\partial}

\newcommand\half{\frac 12}
\newcommand\shalf{\sfrac12}

\newcommand\skp[2]{\left\langle#1,#2\right\rangle}
\newcommand\mw[1]{\left\langle#1\right\rangle}
\renewcommand \exp[1]{\mathrm e^{#1}}

% Nabla und Kombinationen von Nabla

\renewcommand\div[1]{\skp{\nabla}{#1}}
\newcommand\rot{\nabla\times}
\newcommand\grad[1]{\nabla#1}
\newcommand\laplace{\triangle}
\newcommand\dalambert{\mathop{{}\Box}\nolimits}

%Für komplexe Zahlen

\renewcommand \i{\mathrm i}
\renewcommand{\Im}{\mathop{{}\mathrm{Im}}\nolimits}
\renewcommand{\Re}{\mathop{{}\mathrm{Re}}\nolimits}

%Für Bra-Ket-Notation

\newcommand\bra[1]{\left\langle#1\right|}
\newcommand\ket[1]{\left|#1\right\rangle}
\newcommand\braket[2]{\left\langle#1\left.\vphantom{#1 #2}\right|#2\right\rangle}
\newcommand\braopket[3]{\left\langle#1\left.\vphantom{#1 #2 #3}\right|#2\left.\vphantom{#1 #2 #3}\right|#3\right\rangle}


\renewcommand\thesection{Übung \arabic{section}:}
\renewcommand\thesubsection{\arabic{section}.\arabic{subsection}:}

\title{physik511: Übungsblatt 01}
\author{Lino Lemmer}

\begin{document}
\maketitle
\section{Feinstrukturkonstante}

\begin{align*}
    \sbr{e} &= \si{\coulomb^2} \\
    \sbr{\epsilon_0} &= \si{\ampere\second\per\volt\per\meter} = \si{\coulomb^2 \per\joule \per\meter} \\
    \sbr{\hbar} &= \si{\joule\second} \\
    \sbr{c} &= \si{\meter\per\second} \\
    \sbr\alpha &= \frac {\sbr{e}^2} {\sbr{\epsilon_0} \cdot \sbr{\hbar} \cdot \sbr{c}} \\
               &= \frac {\si{\coulomb^2 \joule \meter \second}}{\si{\coulomb^2 \joule \second \meter}} \\
               &= 1
\end{align*}

\section{Auflösungsvermögen}

\subsection{Röntgenstrahlung}

Da die kleineste auflösbare Struktur nicht kleiner sein kann als die Wellenlänge, muss die Wellenlänge atomarer Größenordnung sein, also $\lambda \le \SI{0.1}{\nano\meter}$.

\subsection{Materiestrahl}

\begin{align*}
    \SI{1}{\femto\meter} &\ge \lambda_\text{DB} = \frac hp \\
    \implies \qquad p &\ge \frac h{\SI{1}{\femto\meter}} \\
    p &\ge \SI{6.626e-19}{\newton\second}
\end{align*}

\section{Energie-Impuls-Beziehung}

\begin{align*}
    M^2c^4 &= E^2 - p^2c^2 \\
    \iff \qquad E^2 &= p^2c^2 + M^2c^4 \\
    \iff \qquad E &= \sqrt{p^2c^2 + M^2c^4} \\
                  &= Mc^2 \sqrt{1+\frac{p^2}{M^2c^2}}
    \intertext{Durch Taylor-Entwicklung um \num{1}:}
                  &= Mc^2 \del{1+\half\frac{p^2}{M^2c^2}} \\
                  &= Mc^2 + \frac{p^2}{2M}
    \intertext{Da man die kinetische Energie aus $E - Mc^2$ erhält, ergibt sich}
    E_\text{kin} &= \frac{p^2}{2M}
\end{align*}

\end{document}
